\documentclass[12pt, twoside]{article}
\usepackage[utf8]{inputenc}
\usepackage[english,russian]{babel}

\usepackage{amsthm}
\usepackage{a4wide}
\usepackage{graphicx}
\usepackage{caption}
\usepackage{amssymb}
\usepackage{amsmath}
\usepackage{mathrsfs}
\usepackage{euscript}
\usepackage{graphicx}
\usepackage{subfig}
\usepackage{caption}
\usepackage{color}
\usepackage{bm}
\usepackage{tabularx}
\usepackage{adjustbox}


\usepackage[toc,page]{appendix}

\usepackage{comment}
\usepackage{rotating}

\DeclareMathOperator*{\argmax}{arg\,max}
\DeclareMathOperator*{\argmin}{arg\,min}

\newtheorem{theorem}{Теорема}
\newtheorem{lemma}[theorem]{Лемма}
\newtheorem{definition}{Определение}[section]

\numberwithin{equation}{section}

\newcommand*{\No}{No.}
\begin{document}

\title{\bf Задание априорных знаний на шлюзовую функцию \thanks{Работа выполнена при поддержке РФФИ и правительства РФ.}}
\date{}
\author{}
\maketitle

\begin{center}
\bf
А.\,В.~Грабовой\footnote{Московский физико-технический институт, grabovoy.av@phystech.edu}, В.\,В.~Стрижов\footnote{Московский физико-технический институт, strijov@ccas.ru}

\end{center}

{\centering\begin{quote}
\textbf{Аннотация:} 
\smallskip
\textbf{Ключевые слова}: смесь экспертов; байесовский выбор модели; априорное распределение.

\smallskip
\textbf{DOI}: 00.00000/00000000000000
\end{quote}
}

\section{Введение}
\section{Постановка задачи}
\[
\label{eq:st:6}
\begin{aligned}
p\bigr(y|\textbf{x}, \textbf{W}\bigr)=\lambda\sum_{k=1}^{K}\pi_{k}\bigr(\textbf{x}, \textbf{V}\bigr)p_k\bigr(y|\textbf{x}, \textbf{w}_k\bigr)+\left(1-\lambda\right)\sum_{k=1}^{K}\varepsilon_{k}\bigr(\textbf{x}\bigr)p_k\bigr(y|\textbf{x}, \textbf{w}_k\bigr), \quad \sum_{k=1}^{K}\varepsilon_k\bigr(\textbf{x}\bigr)=1
\end{aligned}
\]
где~$K$~--- количество моделей,~$\textbf{W}$~--- параметры локальных моделей,~$V$~--- параметры шлюзовой функции.

\begin{thebibliography}{99}
\bibitem{Yuksel2012}
	\textit{Yuksel Seniha Esen, Wilson Joseph N., Gader Paul D} Twenty Years of Mixture of Experts~// IEEE Transactions on Neural Networks and Learning Systems. 2012. Issues. 23, No 8. pp. 1177--1193.
		
\bibitem{bishop2006}
	\textit{Bishop C.} Pattern Recognition and Machine Learning.~---~Berlin: Springer, 2006. 758~p.
 \end{thebibliography}


\end{document}

