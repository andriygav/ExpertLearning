\documentclass{lomonosov}
\begin{thesis} 
\Title{Распознавание фигур на изображении с помощью ансамбля моделей}{{Базарова\,А.\,И.}{Грабовой\,А.\,В.}{Стрижов\,В.\,В.}}
\AuthorM{{Базарова~Александра~Ильинична}{Грабовой~Андрей~Валериевич}{Стрижов~Вадим~Викторович}}{
	{Студент, ФУПМ МФТИ, Долгопрудный, Россия}{Студент, ФУПМ МФТИ, Долгопрудный, Россия}{Научный сотрудник, Вычислительный центр им. А. А. Дородницына Федерального исследовательского центра <<Информатика и управление>> Российской академии наук, Москва, Россия}}{bazarova.ai@phystech.edu,  grabovoy.av@phystech.edu, strijov@phystech.edu}

В данной работе решается задача распознавания заданного набора фигур на изображении в предположении, что фигуры являются кривыми второго порядка. Построение моделей машинного обучения основывается на информации о виде этих кривых и множестве их возможных преобразований. Такую информацию называют \textit{экспертными знаниями}, а метод машинного обучения, основанный на \textit{экспертных знаниях}, называют \textit{обучением с экспертом}.

    В работе предлагается отобразить точки, принадлежащие кривым второго порядка, в новое признаковое пространство, в котором каждая кривая второго порядка аппроксимируется линейной моделью, называемой локальной. При распознавании нескольких кривых на одном изображении на основании локальных моделей строится мультимодель, называемая смесью экспертов. Эта мультимодель взвешивает локальные модели с помощью шлюзовой функции и  аппроксимирует выборку. Значения весовых коэффициентов зависят от того объекта, для которого производится предсказание. Таким образом, набор кривых высших порядков распознается при помощи композиции линейных моделей. В работе поставлена и решена задача оптимизации параметров мультимодели.
    
    Качество работы предложенного метода сравнивается на синтетических данных и на реальных изображениях, которые включают в себя кривые второго порядка.
    
    \begin{references}
\Source \ENGLISH{Graboviy\,A.\,V., Strijov\,V.\,V. Analysis of prior distributions for a mixture of experts // Computational Mathematics and Mathematical Physics, to appear in 2020.}
\Source \ENGLISH{Scheres\,Sjors\,H.\,W. A Bayesian view on Cryo-EM structure determination. // Journal of Molecular Biology. 2012. Vol.415. \No 2. p. 406--418.} 
\Source \ENGLISH{Yuksel\,S.\,E., Wilson\,J.\,N., Gader\,P.\,D. Twenty years of mixture of experts // IEEE Transactions on Neural Networks and Learning Systems. 2012. Vol. 23, \No 8, p. 1177--1193.}

\Source \ENGLISH{I.\,A.\,Matveev. Detection of iris in image by interrelated maxima of brightness gradient projections //  Applied and Computational Mathematics. 2010. Vol. 9. \No 2. p. 252--257. }



\end{references}
\end{thesis}