%
%   Образец / Шаблон оформления тезиса
%
%
%   Если в тезисе каких-то разделов (картинок, списка литературы) нет, то соотвествующие команды надо закомментировать.
%   Файл для компиляции --- этот (example.tex, переименовый в фамилию автора, например, ivanov.tex).
%
%   ========================================================================================
%


%
%   Если в вашем документе нет картинок и вы хотите компилировать документ при помощи latex->dvips->ps2pdf, то уберите опцию usePics, заменив следующую строчку на
\documentclass{lomonosov}
%\documentclass[usePics]{lomonosov}

\begin{thesis}  % Сам тезис должен быть полностью помещен внутри окружения thesis

% Один автор
%\Title{Тема доклада}{{Иванов\,И.\,И.}}
% Несколько авторов
\Title{Распознавание фигур на изображении с помощью ансамбля моделей}{{Базарова\,А.\,И.}{Грабовой\,А.\,В.}}

%
%   Команда авторства. Выберете ту, что отвечает вашему тезису, и, если надо, раскомментируйте ее; остальные --- удалите или закомментируйте.
%

% Один автор
%\Author{Иванов~Иван~Иванович}{Студент}{Факультет ВМК МГУ имени М.\,В.\,Ломоносова}{Москва}{Россия}{ivanov@cmc.msu.ru}

% Несколько авторв из одной организации
\Author{Базарова~Александра~Ильинична, Грабовой~Андрей~Валериевич}{Студентка бакалавриата, студент магистратуры}{ФУПМ МФТИ}{Долгопрудный}{Россия}{bazarova.ai@phystech.edu, grabovoy.av@phystech.edu}{Стрижов Вадим Викторович}

% Несколько авторов из разных организаций
%\AuthorM{{Иванов~Иван~Иванович}{Петров~Петр~Петрович}}{%
%   {Аспирант, факультет ВМК МГУ имени М.\,В.\,Ломоносова, Москва, Россия}{Младший научный сотрудник, Ленинградский кораблестроительный институт, Ленинград, СССР}}{ivanov@cmc.msu.ru, petrov@cmc.msu.su}

В данной работе решается задача распознавания заданного набора фигур на изображении в предположении, что фигуры являются кривыми второго порядка. Построение моделей машинного обучения основывается на информации о виде этих кривых и множестве их возможных преобразований. Такую информацию называют \textit{экспертными знаниями}, а метод машинного обучения, основанный на \textit{экспертных знаниях}, называют \textit{обучением с экспертом} [1, 3].

    В работе предлагается отобразить точки, принадлежащие кривым второго порядка, в новое признаковое пространство, в котором каждая кривая второго порядка аппроксимируется линейной моделью, называемой локальной. При распознавании нескольких кривых на одном изображении на основании локальных моделей строится мультимодель, называемая смесью экспертов [3]. Эта мультимодель взвешивает локальные модели с помощью шлюзовой функции и  аппроксимирует выборку. Значения весовых коэффициентов зависят от того объекта, для которого производится предсказание. Таким образом, набор кривых высших порядков распознается при помощи композиции линейных моделей. В работе поставлена и решена задача оптимизации параметров мультимодели.
    
    Качество работы предложенного метода сравнивается на синтетических данных и на реальных изображениях, которые включают в себя кривые второго порядка [4].

%
%   Список литературы, если он есть
%
\begin{references}
\Source \ENGLISH{Graboviy\,A.\,V., Strijov\,V.\,V. Analysis of prior distributions for a mixture of experts // Computational Mathematics and Mathematical Physics, to appear in 2020.}
\Source \ENGLISH{Scheres\,S.\,H.\,W. A Bayesian view on Cryo-EM structure determination. // Journal of Molecular Biology. 2012. Vol.\,415. \No\,2. P. 406--418.} 
\Source \ENGLISH{Yuksel\,S.\,E., Wilson\,J.\,N., Gader\,P.\,D. Twenty years of mixture of experts // IEEE Transactions on Neural Networks and Learning Systems. 2012. Vol.\,23, \No\,8, P. 1177--1193.}

\Source \ENGLISH{Matveev\,I.\,A. Detection of iris in image by interrelated maxima of brightness gradient projections //  Applied and Computational Mathematics. 2010. Vol.\,9. \No\,2. P. 252--257. }
\end{references}

\end{thesis} % Сам тезис должен быть полностью помещен внутри окружения thesis
