\documentclass[12pt, twoside]{article}
\usepackage[utf8]{inputenc}
\usepackage[english,russian]{babel}

\usepackage{amsthm}
\usepackage{a4wide}
\usepackage{graphicx}
\usepackage{caption}
\usepackage{amssymb}
\usepackage{amsmath}
\usepackage{mathrsfs}
\usepackage{euscript}
\usepackage{graphicx}
\usepackage{subfig}
\usepackage{caption}
\usepackage{color}
\usepackage{bm}
\usepackage{tabularx}
\usepackage{adjustbox}


\usepackage[toc,page]{appendix}

\usepackage{comment}
\usepackage{rotating}

\DeclareMathOperator*{\argmax}{arg\,max}
\DeclareMathOperator*{\argmin}{arg\,min}

\newtheorem{theorem}{Теорема}
\newtheorem{lemma}[theorem]{Лемма}
\newtheorem{definition}{Определение}[section]

\numberwithin{equation}{section}

\newcommand*{\No}{No.}
\begin{document}

\title{\bf Анализ свойств моделей в задачах обучения с экспертом}
\date{}
\author{}
\maketitle

\begin{center}
\bf
А.\,И. Базарова\footnote{Московский физико-технический институт, bazarova.ai@phystech.edu}, А.\,В.~Грабовой\footnote{Московский физико-технический институт, grabovoy.av@phystech.edu}, В.\,В.~Стрижов\footnote{Московский физико-технический институт, strijov@ccas.ru}

\end{center}

{\centering\begin{quote}
\textbf{Аннотация:} В данной работе решается задача поиска заданного набора фигур на изображении в предположении, что фигуры являются кривыми второго порядка. Построение моделей машинного обучения основывается на информации о виде этих кривых и множестве их возможных преобразований. Такую информацию называют \textit{экспертными знаниями}, а метод машинного обучения, основанный на \textit{экспертных знаниях}, называют \textit{обучением с экспертом}.
    В работе предлагается отобразить кривые второго порядка в новое признаковое пространство, в котором каждая локальная модель является линейной моделью. Таким образом, распознавание кривых высших порядков может быть осуществлено при помощи композиции линейных моделей. В работе поставлена и решена задача оптимизации для поиска оптимальных параметров мультимодели.
    Качество работы предложенного метода сравнивается на синтетических данных и датасетах с реальными изображениями, которые включают в себя кривые второго порядка.
    
\smallskip
\textbf{Ключевые слова}: смесь моделей, обучение с экспертом, линейные модели

\smallskip
\textbf{DOI}: 00.00000/00000000000000
\end{quote}
}

\end{document}

