\documentclass[twoside]{article}
\usepackage{idp20}

\begin{document}

\Russian
\title{Задача обучения с экспертом для построение интерпретируемых моделей машинного обучения}
\author{Грабовой~А.\,В.}{Грабовой Андрей Валериевич$^{1}$\speaker}{grabovoy.av@phystech.edu}
\author{Стрижов~В.\,В.}{Стрижов Вадим Викторович$^{1,2}$}{strijov@phystech.edu}
\organization{%
    $^1$Москва, Московский физико-технический институт\par
    $^2$Москва, Вычислительный центр им. А.\,А.~Дородницына ФИЦ ИУ РАН}
\maketitle

Построение интерпретируемых моделей является одной из основных проблем в машинном обучении. Для повышения качества аппроксимации повышается сложность модели, из-за чего снижается ее интерпретируемость. Для повышения качества аппроксимации без повышения сложности модели предлагается использовать экспертную информацию о данных. Метод машинного обучения, основанный на экспертной информации, называется \textit{обучением с экспертом}.

Данное исследование посвящено построению интерпретируемых моделей на основе экспертного априорного представления о решаемой задаче. Решается задача аппроксимации кривых второго порядка на контурном изображении. Предполагается, что кривая второго порядка описывается одной моделью. Экспертная информация позволяет отобразить точки изображения в новое пространство, где данная кривая описывается линейной моделью. При аппроксимации нескольких кривых на одном изображении строится мультимодель на основе смеси экспертов.

Решается задача аппроксимации радужки глаза. Глаз представляется как две концентрические окружности на изображении. Каждая окружность аппроксимируется одной линейной моделью. Изображение глаза является набором точек, которые требуется аппроксимировать. В вычислительном эксперименте анализируется качество аппроксимации контурного изображения при помощи предложенного метода. Проводится анализ качества аппроксимации радужки глаза. 

Работа выполнена при поддержке РФФИ (проекты 19-07-01155, 19-07-00875) и НТИ (проект 13/1251/2018).

\begin{thebibliography}{1}
\bibitem{grabovoy2020}
    \emph{Grabovoy~A.  Strijov~V.}
    Prior distribution choices for a mixture of expert~// Computational Mathematics and Mathematical Physics, 2020.
\end{thebibliography}

\English
\title{Expert learning for interpretable model selection}
\author{Grabovoy~A.}{Andrey Grabovoy$^1$\speaker}{grabovoy.av@phystech.edu}
\author{Strijov~V.}{Vadim Strijov$^{1}$}{strijov@phystech.edu}
\organization{%
    $^1$Moscow, Moscow Institute of Physics and Technology\par
    $^2$Moscow, FRCCSC of the Russian Academy of Sciences}
\maketitle
The interpretable model selection is an important problem in machine learning. To improve the approximation quality one has to increase the number of model parameters. It results in a less interpretable model. The authors propose to use an expert prior information to improve the quality without increasing the complexity of the model. This information is called the expert information. The machine learning method based on an expert information is called expert learning.

This work analyzes the interpretable model selection. It solves the problem of approximating second-order curves on a given contour image.
It is assumed that the second-order curve is described by one model, and the expert information maps the points of an image into a new space, where this curve is described by a linear model.
 A mixture of experts approximate to approximate several curves on one image.

A circle is an example of a second-order curve. The iris is represented as two concentric circles in an image. Each circle is approximated by one linear model.

The computational experiment analyses the proposed method for the contour image approximation. It uses synthetic and real data to test the proposed method. The real data is a human eye image from the iris detection problem.

This research was supported by RFBR (projects 19-07-01155, 19-07-00875) and NTI (project 13/1251/2018).

\begin{thebibliography}{1}
\bibitem{grabovoy2020}
    \emph{Grabovoy~A.  Strijov~V.}
    Prior distribution choices for a mixture of expert~// Computational Mathematics and Mathematical Physics, 2020.
\end{thebibliography}

\end{document} 